% $Id$

This section summarizes the project milestones and the expected outcomes.

\subsection{\label{ssec:plan}Milestones}

Based on Section~\ref{ssec:aims}, we break down time scale needed to develop
each component as belows. The milestones would like more ``conceptual'' rather
than practical or tangible. The project dates listed below are only indicative
as the project will evolve based on the results being obtained while working on
it.

\paragraph{\textsf{May 26}} \underline{\textbf{GSoC Project Start}}

\paragraph{\textsf{May 31}} Finish background research and readings: \emph{vic} source
codes and UltraGrid's TFRC implementation along with RTP extension part.  The
aim here is try looking at how TFRC was implemented over a \emph{vic}-like platform
(different library and different language so it would not be possible to
directly re-use them.).

\paragraph{\textsf{Jun. 6}} Commit necessary TFRC files - e.g., TFRC
(\texttt{tfrc\_sndr.cpp, tfrc\_rcvr.cpp}, etc), RTP integration
(\texttt{rtp\_tfrc.cpp}, etc), CC handler (\texttt{cc.cpp}, etc). It would not
be necessariliy a full implementation at this stage yet.

\paragraph{\textsf{Jun. 20}} Primitive TFRC implementation. At this stage, we
should be able to send a packet using TFRC (with dummy packets).

\paragraph{\textsf{Jun. 27}} Commit necessary TFWC files. It would not be
necessariliy a full implementation at this stage yet. 

\vspace{.5cm}

\begin{center}
\textsf{the primitive TFRC commit \emph{not} made yet as of July 1. 
\\ the immediate task should be designing CC APIs as early as possible.}
\end{center}

\vspace{.5cm}

\paragraph{\textsf{Jul. 4}} Primitive TFWC implementation

\paragraph{\textsf{Jul. 7}} \underline{\textbf{Mid-term Evaluation}} -- Write
interim report: what has been obtained successfully and what has not, what
should be changed for the final project goal, etc. We need a detailed CC API
design document by this date.

\paragraph{\textsf{Jul. 18}} Implement a sending buffer and modify the
\emph{vic}'s transmit module by this date. Preferably, the draft version of
TFR(W)C implementation (i.e., doesn't have to do much things, but the
fundamental function calls should be done by this date).  Examine what feedback
information can be used to codecs.  (e.g., modify codecs?  or use Q-factor in a
specific codec?, etc)

\paragraph{\textsf{Jul. 25}} Add detailed TFR(W)C functionality (i.e., basic CC
mechanisms should be done by this date).

\paragraph{\textsf{Aug. 1}} Start looking at the interactions between CC modules
and codecs. Preferably, the CC modules have been valdiated (i.e., correct
feedback information coming in and going out in case of TFRC, or \emph{cwnd}
size working fine as the algorithm? etc.)

\paragraph{\textsf{Aug. 8}} Start putting all together. The suggest \emph{pencil
down} date is the $11^{th}$ of August. Finalizing the rest of the codes.

\paragraph{\textsf{Aug. 15}} Wrap up and write-up.

\paragraph{\textsf{Aug. 18}} \underline{\textbf{Project Submission}}-- Submit
the results to Google and wrap-up. Final evaluation. 

\paragraph{\textsf{Sept. 1}} Submit the report and codes to Google.


\subsection{\label{ssec:outcomes}Expected Outcomes}

\subsubsection{\label{sssec:codes}Program Source Code}

\begin{itemize} 
\item Congestion Control Modules (TFRC and TFWC) 
\item RTP Interface with CC Modules 
\item Video codec Enhancement 
\end{itemize}

\subsubsection{\label{sssec:docs}Reports}

\begin{itemize}
\item Interim Report
\item Final Report with Results
\end{itemize}

\newpage
