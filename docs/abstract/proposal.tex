% THIS IS SIGPROC-SP.TEX - VERSION 3.0
% WORKS WITH V3.1SP OF ACM_PROC_ARTICLE-SP.CLS
% JUNE 2007
%
% It is an example file showing how to use the 'acm_proc_article-sp.cls' V3.1SP
% LaTeX2e document class file for Conference Proceedings submissions.
% ----------------------------------------------------------------------------------------------------------------
% This .tex file (and associated .cls V3.1SP) *DOES NOT* produce:
%       1) The Permission Statement
%       2) The Conference (location) Info information
%       3) The Copyright Line with ACM data
%       4) Page numbering
% ---------------------------------------------------------------------------------------------------------------
% It is an example which *does* use the .bib file (from which the .bbl file
% is produced).
% REMEMBER HOWEVER: After having produced the .bbl file,
% and prior to final submission,
% you need to 'insert'  your .bbl file into your source .tex file so as to provide
% ONE 'self-contained' source file.
%
% Questions regarding SIGS should be sent to
% Adrienne Griscti ---> griscti@acm.org
%
% Questions/suggestions regarding the guidelines, .tex and .cls files, etc. to
% Gerald Murray ---> murray@acm.org
%
% For tracking purposes - this is V3.0SP - JUNE 2007

\documentclass{style}

\begin{document}

\title{Congestion Control for Real-time Interactive Applications 
\textsf{TFR(W)C over VIC}
%\subtitle{\textsf{TFR(W)C over VIC}
\titlenote{This idea was submitted to \textit{Google Summer of Code 2008} -
\textsf{http://code.google.com/soc/2008}}
\vspace{-0.2in}
}
%
% You need the command \numberofauthors to handle the 'placement
% and alignment' of the authors beneath the title.
%
% For aesthetic reasons, we recommend 'three authors at a time'
% i.e. three 'name/affiliation blocks' be placed beneath the title.
%
% NOTE: You are NOT restricted in how many 'rows' of
% "name/affiliations" may appear. We just ask that you restrict
% the number of 'columns' to three.
%
% Because of the available 'opening page real-estate'
% we ask you to refrain from putting more than six authors
% (two rows with three columns) beneath the article title.
% More than six makes the first-page appear very cluttered indeed.
%
% Use the \alignauthor commands to handle the names
% and affiliations for an 'aesthetic maximum' of six authors.
% Add names, affiliations, addresses for
% the seventh etc. author(s) as the argument for the
% \additionalauthors command.
% These 'additional authors' will be output/set for you
% without further effort on your part as the last section in
% the body of your article BEFORE References or any Appendices.

\numberofauthors{1} %  in this sample file, there are a *total*
% of EIGHT authors. SIX appear on the 'first-page' (for formatting
% reasons) and the remaining two appear in the \additionalauthors section.
%
\author{
% You can go ahead and credit any number of authors here,
% e.g. one 'row of three' or two rows (consisting of one row of three
% and a second row of one, two or three).
%
% The command \alignauthor (no curly braces needed) should
% precede each author name, affiliation/snail-mail address and
% e-mail address. Additionally, tag each line of
% affiliation/address with \affaddr, and tag the
% e-mail address with \email.
%
% 1st. author
\alignauthor
Soo-Hyun Choi\\
       \affaddr{Department of Computer Science}\\
       \affaddr{University College London}\\
       \affaddr{Malet Place, London, WC1E 6BT}\\
       \email{s.choi@cs.ucl.ac.uk}
}
% There's nothing stopping you putting the seventh, eighth, etc.
% author on the opening page (as the 'third row') but we ask,
% for aesthetic reasons that you place these 'additional authors'
% in the \additional authors block, viz.
%\additionalauthors{Additional authors: John Smith (The Th{\o}rv{\"a}ld Group,
%email: {\texttt{jsmith@affiliation.org}}) and Julius P.~Kumquat
%(The Kumquat Consortium, email: {\texttt{jpkumquat@consortium.net}}).}
\date{April 05, 2008}
% Just remember to make sure that the TOTAL number of authors
% is the number that will appear on the first page PLUS the
% number that will appear in the \additionalauthors section.

\maketitle
\begin{abstract}
\input{abstract}
\end{abstract}

%\vspace{-0.1in}
% A category with the (minimum) three required fields
%\category{C.2.2}{Computer-Communication Networks}{Network Protocols}
%A category including the fourth, optional field follows...
%\category{C.2.6}{Computer-Communication Networks}{Internetworking}[Standards
%(e.g., TCP/IP)]

%\vspace{-0.1in}
%\terms{Design, Performance}

%\vspace{-0.1in} 
%\keywords{congestion control, transport protocols, real-time streaming media,
%TFRC, TCP}

\section{Introduction}
% $Id$

In recent years we have seen a great increase in networked multimedia
applications; many of them greatly prefer to use UDP for transmission because
the variability of TCP's congestion control mechanism and its complete
reliability do not match their timeliness requirements. These applications, such
as streaming media, Internet telephony, and games have in common two key demands
of a transport protocol:

\vspace{-0.2in}
\begin{itemize}

\item timely packet delivery
\item smooth predictable transmission rate

\end{itemize}

\vspace{-0.2in}
Proposed congestion control mechanisms to achieve these characteristics include
generalizations of TCP-like window-based schemes (e.g., GAIMD~\cite{YL00},
TEAR~\cite{ROY00}, Binomial TCP~\cite{BB01}) and equation-based congestion
control scheme (e.g., TFRC~\cite{FHPW00}).  In this work for GSoC 2008, we would
like to implement the latter cases: TFRC proposed by Floyd~\cite{FHPW00} and
TFWC proposed by Soo-Hyun~\cite{SH06} over VIC/RAT audio/video
applications~\cite{AVATS} using \textsf{C/C++}.



\section{Motivation}
% $Id$

\subsection{does congestion control ever needed?}

One might be able to think that a real-time interactive streaming application
does not ever need the congestion control mechanisms, but in fact we can do much
things using congestion control over those applications. The strong part of the
contribution is that we get the precise control about the end-to-end packet
delay in any kind of networks: congestion control mechanisms can provide the
relevant information to upper layer (e.g., multimedia codecs). Considering the
fact that the delay (and delay jitter) is one of the critical parameters that a
user might greatly care specially for the ``interactive" multimedia streaming
applications, the controlled delay using congestion control could bring a
significant advantage to such applications.

\subsection{then, why TFRC and why TFWC?}

As we pointed out already, our target applications prefer smooth and predictable
transmission rate to a TCP-like rate changing behavior. To achieve this goal,
Sally proposed TFRC~\cite{FHPW00}, a de facto standard, in 2000 and it is
gaining a great popularity: it is also adopted DCCP's CCID3, being standardized
by IETF.  However, we have recently observed that TFRC has some limitations such
that:

\vspace{-0.2in}
\begin{itemize}

\item It can produce uncontrolled throughput oscillation in a certain network
conditions~\cite{CH07}.

\item The TCP throughput equation has the convexity property which
resulting throughput imbalance~\cite{RX07}.

\item The difficulty of measuring RTT correctly can result in a long-term 
throughput imbalance~\cite{AS06}.

\end{itemize}

\vspace{-0.2in} 
All of the problems above essentially stem from the same basic cause: it is
difficult to build a rate-based protocol with a rate that is directly inverse
proportional to RTT while achieving stability in all conditions. All these
issues beg the question of why TFRC is rate-based at all?  We have proposed a
window-based version of TFRC, which uses a TCP-like Ack-clocking feature but
merges this with the use of the TCP throughput equation, ala TFRC, to directly
adjust the sending window size. The goal is to remedy the issues discussed above
which relate to the combination of rate-based control. Using a window makes the
RTT implicit in the Ack clock, and removing the need to be rate-based makes life
much simpler for application writers, as they no longer need to work around the
limitations of the OS's short duration timers. The detailed protocol description
and initial results can be found at~\cite{SH06}.

In next section, we describe our implementation proposal for GSoC 2008.


\section{Detailed Proposal and Plan}
% $Id$

\subsection{Overview}

First, we plan to implement TFRC on VIC using C language.  Given that there is a
similar TFRC implementation [6] over UltraGrid using the UCL common library, we
could leverage their implementation as VIC also uses the same library.  This
includes the APIs with TFRC and RTP/RTCP and some front-end congestion control
options. The reason for the choice of VIC is that the size of video packet is
much larger than that of voice, so we could test larger packet first and then
move on to the smaller size like RAT.

Second, we plan to implement TFWC from the scratch using C. There is a
simulation code implemented on ns-2~\cite{ns2} using C++, and we hope to use it
again here in VIC. But, we will need to implement from the scratch for its APIs
with RTP/RTCP and integrate with VIC's user interfaces.

\subsection{Components to be developed}

\begin{itemize}

\item TFRC sender/receiver: implement algorithm as described in~\cite{FHPW00}.

\item TFWC sender/receiver: implement algorithm as described in~\cite{SH06}.

\item Send Buffer: it is critical to implement send buffer carefully as
   this is the main part to make interact correctly with our congestion
   control modules. The send buffer should be able to handle RTT calculation,
   update e2e delay, maintain queue, maintain frame length, and finally compute
   the inter-packet-interval.

\item Receive Buffer: the play-out buffer is rather straight forward as it only
   needs to play the packet streams.

\item Congestion Control (CC) Manager: main CC interfaces with RTP/RTCP. update
   CC's sender/receiver state along with necessary parameters.

\item APIs: TFR(W)C's APIs in order for the communication with RTP/RTCP. It is
   necessary to make sure that our congestion control modules give/get
   information to/from RTP layer correctly with newly created APIs.

\end{itemize}

\vspace{-0.2in}
Apart from the above components, we will need to change quite a bit of VIC main
source code to fit our modules.


\section{Miscellaneous}
% $Id$

\textsf{\textbf{why I am a strong candidate?}} Implementing a network protocol
can be sometimes challenging task as it is often hard to validate every function
in every aspect.  However, I already developed TFWC on ns-2 and know exact steps
how to develop such congestion control protocols.  Also, AVATS project is now
being conducted in the same department that I'm working on, and I know well the
PI and the main developer as we are just sitting next to each other.  My current
research is also based on the part of their software.  This could be a natural
adaptation for us to start co-working, meaning that we could achieve the goal in
a relatively short time scale with my experience, and it also could be easier to
co-work because we are geographically in the same place.



%ACKNOWLEDGMENTS are optional
%\section{Acknowledgments}
%\input{acknowledgments}

%
% The following two commands are all you need in the
% initial runs of your .tex file to
% produce the bibliography for the citations in your paper.
\small
\bibliographystyle{abbrv}
% reference.bib is the name of the Bibliography in this case
\bibliography{reference}

% You must have a proper ".bib" file
%  and remember to run:
% latex bibtex latex latex
% to resolve all references
%
% ACM needs 'a single self-contained file'!
%
%APPENDICES are optional
%\balancecolumns
%\appendix
%\input{appendix}
% That's all folks!
\end{document}
