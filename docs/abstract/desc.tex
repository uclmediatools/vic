% $Id$

\subsection{Overview}

First, we plan to implement TFRC on VIC using C language.  Given that there is a
similar TFRC implementation [6] over UltraGrid using the UCL common library, we
could leverage their implementation as VIC also uses the same library.  This
includes the APIs with TFRC and RTP/RTCP and some front-end congestion control
options. The reason for the choice of VIC is that the size of video packet is
much larger than that of voice, so we could test larger packet first and then
move on to the smaller size like RAT.

Second, we plan to implement TFWC from the scratch using C. There is a
simulation code implemented on ns-2~\cite{ns2} using C++, and we hope to use it
again here in VIC. But, we will need to implement from the scratch for its APIs
with RTP/RTCP and integrate with VIC's user interfaces.

\subsection{Components to be developed}

\begin{itemize}

\item TFRC sender/receiver: implement algorithm as described in~\cite{FHPW00}.

\item TFWC sender/receiver: implement algorithm as described in~\cite{SH06}.

\item Send Buffer: it is critical to implement send buffer carefully as
   this is the main part to make interact correctly with our congestion
   control modules. The send buffer should be able to handle RTT calculation,
   update e2e delay, maintain queue, maintain frame length, and finally compute
   the inter-packet-interval.

\item Receive Buffer: the play-out buffer is rather straight forward as it only
   needs to play the packet streams.

\item Congestion Control (CC) Manager: main CC interfaces with RTP/RTCP. update
   CC's sender/receiver state along with necessary parameters.

\item APIs: TFR(W)C's APIs in order for the communication with RTP/RTCP. It is
   necessary to make sure that our congestion control modules give/get
   information to/from RTP layer correctly with newly created APIs.

\end{itemize}

\vspace{-0.2in}
Apart from the above components, we will need to change quite a bit of VIC main
source code to fit our modules.
